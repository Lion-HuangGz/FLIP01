% 
% ---------------------------------------------------------------
% Copyright (C) 2012-2018 Gang Li
% ---------------------------------------------------------------
%
% This work is the default powerdot-tuliplab style test file and may be
% distributed and/or modified under the conditions of the LaTeX Project Public
% License, either version 1.3 of this license or (at your option) any later
% version. The latest version of this license is in
% http://www.latex-project.org/lppl.txt and version 1.3 or later is part of all
% distributions of LaTeX version 2003/12/01 or later.
%
% This work has the LPPL maintenance status "maintained".
%
% This Current Maintainer of this work is Gang Li.
%
%

\documentclass[
 size=12pt,
 paper=smartboard, %a4paper, smartboard, screen
 mode=present, %present, handout, print
 display=slides, % slidesnotes, notes, slides
% nohandoutpagebreaks,
% pauseslide,
style=tuliplab,
% nopagebreaks,clock
% hlentries=true,
% hlsections = true,
pauseslide,
fleqn,leqno]{powerdot}

\hypersetup{pdfpagemode=FullScreen}
% \usepackage[toc,highlight,blackslide,slidesonly,sounds,HA]{HA-prosper}

\usepackage{amssymb}
\usepackage{amsmath} 
\usepackage{rotating}
\usepackage{graphicx}
\usepackage{boxedminipage}
\usepackage{media9}
\usepackage{rotate}
\usepackage{calc}
\usepackage[absolute]{textpos}
\usepackage{psfrag,overpic}
\usepackage{fouriernc}
\usepackage{pstricks,pst-node,pst-text,pst-3d,pst-grad}
\usepackage{moreverb,epsfig,color,subfigure}
\usepackage{color}
\usepackage{pstricks}
\usepackage{pstricks-add}
\usepackage{pst-text}
\usepackage{pst-node, pst-tree}
\usepackage{booktabs}
\usepackage{etex}
\usepackage{breqn}
\usepackage{multirow}
\usepackage{gitinfo2}

\usepackage{listings}
\lstset{frameround=fttt, 
frame=trBL, 
stringstyle=\ttfamily,
backgroundcolor=\color{yellow!20},
basicstyle=\footnotesize\ttfamily}
\lstnewenvironment{code}{
\lstset{frame=single,escapeinside=`',
backgroundcolor=\color{yellow!20},
basicstyle=\footnotesize\ttfamily}
}{}


\usepackage{fouriernc}
\usepackage{hyperref}

%%%%%%%%%%%%%%%%%%%%%%%%%%%%%%%%%%%%%%%%%%%%%%%%%%%%%%%%%%%%%%%%%%%%%%%%
% title
% TODO: Customize to your Own Title, Name, Address
%
\title{FLIP01 FINAL PRESENTATION}
\author{
Guanzhang Huang
\\
Xi'an Shiyou University 
% \href{mailto:gangli@acm.org}{gangli@acm.org}
% \and % more authors
}
\date{\today}


% Customize the setting of slides
\pdsetup{
% theslide=\arabic{slide}~/~\pageref*{lastslide},
% theslide=\arabic{slide},
rf=\href{http://www.tulip.org.au}{
Last Changed by: \textsc{\gitCommitterName}\ \gitVtagn-\gitAbbrevHash\ (\gitAuthorDate)
},
cf={FLIP01 FINAL PRESENTATION},
%trans=Fade,
%list={labelsep=1em,leftmargin=*,itemsep=0pt,topsep=5pt,parsep=0pt},
% counters={theorem,lemma},
% randomdots,dmaxdots=80
}


\begin{document}

\maketitle 
\begin{slide}[toc=,bm=]{Overview}
  \tableofcontents[content=sections]
\end{slide}

  \section{Problem Statement}

  \begin{slide}{Problem Definition}
  %\tableofcontents[content=currentsection,type=1]
 \hspace{0.5cm}  Using 8 years daily news headlines to predict stock market movement
  \end{slide}
  %\begin{slide}{The overview of the question }
    %\vspace{2cm}
    %\setlength{\parindent}{1.5em}
    %You are given 5 years of store-item sales data, and asked to predict 3 months of sales for 50 different items at 10 different stores.
  %\end{slide}
  \begin{slide}{Data Set}
  \begin{itemize}


    \item data information
    \begin{table}[htbp]  \centering
      \caption{The head of the data}
      \label{tbl:data information}
      \begin{tabular}{ccccccc}
        \hline
        % after \\: \hline or \cline{col1-col2} \cline{col3-col4} ...
        & Date & label & Top1 & ...\\
        \hline
        0 & 2008-08-08    & 0     & b"Georgia 'downs two Russian warplanes' as cou... & ...\\
        1 & 2008-08-11    & 1     & b'Why wont America and Nato help us? If they w... & ...\\
        2 & 2008-08-12    & 0     & b'Remember that adorable 9-year-old who sang a... & ...\\
        3 & 2008-08-13    & 0     & b' U.S. refuses Israel weapons to attack Iran:... & ...\\
        4 & 2008-08-14    & 1     & b'All the experts admit that we should legalis... & ...\\
        5 & ...           & ...   & ...                                               & ...\\
        \hline 
        %\bottomrule
      \end{tabular}
    \end{table}
    \

  %\item Display the data set 
  %\begin{table}[htbp]  \centering
  %  \caption{The head of the test data}
  %  \label{tbl:data information}
  %  \begin{tabular}{ccccccc}
      % after \\: \hline or \cline{col1-col2} \cline{col3-col4} ...
  %     & id & ingredients\\
  %    \hline
  %    0  & 18009 & [baking powder, eggs, all-purpose flour, raisi... \\
  %    1  & 28583 & [sugar, egg yolks, corn starch, cream of tarta... \\
  %    2  & 41580 & [sausage links, fennel bulb, fronds, olive oil... \\
  %    3  & 29752 & [meat cuts, file powder, smoked sausage, okra,... \\
  %    4  & 35687 & [ground black pepper, salt, sausage casings, l... \\
  %    \hline 
      %\bottomrule
  %  \end{tabular}
  %\end{table}
  \end{itemize}
\end{slide}
 % \begin{slide}[toc=,bm=]{Overview}
    %\tableofcontents[content=sections]
    %\end{slide}
    %\section{Second section}
    %\begin{slide}[toc=,bm=]{Main contents}
    %\tableofcontents[content=currentsection,type=1]
    %\end{slide}
    %\begin{slide}{Problem analysis by using data visualization}
      %\begin{itemize}
      %\item List the directories and files and load  data set
      %\item The introduction of the data set
      %\item Plot statistical charts and see the sale pattern
      %\item Variation in scale of the sale transacted
      %\item Store total sales
      %\item Item total sales
      %\item All store's performance
      %\item Individual pattern of store's and item's sales
      %\end{itemize}
    %\end{slide} 
\section{Text Processing}

\begin{slide}{Preprocessing}
%By using stopwords,regularization to process the text\\
\vspace{1cm}
\begin{itemize}
  \item Get rid of the HTML tag
  \item Remove the punctuation
  \item Cut into the word /token
  \item stopwords
  \item Reorganize into new sentences
\end{itemize}
\vspace{1cm}

stop words for example:\\
"'d",  "'ll",  "'m",  "'re",  "'s",  "'t",  "'ve"  "ZT",  "ZZ", "a" ,"a's"  ...
 

\end{slide}



\begin{slide}{Word2vec}
  \item embedding_size\\ %词向量维度
  \item learning_rate\\  %学习率
  \item nagetive_sample\\%负采样模型
\hspace{1.5cm}
%\begin{figure}[ht]%插入图片
%  \centering%用于居中
%  \includegraphics[scale=0.85]{E:/tulip-flip/flip01/photo/01.eps}
%  \includegraphics[scale=0.6]{E:/tulip-flip/flip01/photo/05.eps}
%  \caption{Displaying the words in text}%图片标题
%  \end{figure} 
\end{slide}

\begin{slide}{word2vec}
  %\tableofcontents[content=currentsection,type=1]
 \hspace{0.5cm}  
   \item model[ok]\\
   \vspace{1cm}
   128D:array([-0.29960674,0.03145241...])
  \end{slide}

%\section{Text feature extraction}

%\begin{slide}{output}
\\
%\vspace{1cm}
%\begin{itemize}
%  \item unique() and apply()
%  \item word2vec
%\end{itemize}
%\end{slide}

%\begin{slide}{Replace text labels}
%By using the unique() and apply() can replace the texe label into figures.\\
%\begin{table}[htbp]  \centering
%  \caption{Replace the text label}
%  \label{tbl:data information}
%  \begin{tabular}{ccccccc}
%    % after \\: \hline or \cline{col1-col2} \cline{col3-col4} ...
%    & cuisine  & label\\
%    \hline
%    0  & irish   & 16 \\
%    1  & italian & 6  \\
%    2  & irish   & 16 \\
%    3  & chinese & 8  \\
%    4  & mexican & 7  \\
%    \hline 
%    %\bottomrule
%  \end{tabular}
%\end{table}

%  \vspace{1cm}
%  \end{slide}

%\begin{slide}{word2vec}
%Use word2vec to convert text to word vectors. And convert word vectors to sentence vector.and then for each 
%sentence vector we have one label for it.\\
%\begin{itemize}
%  \item vector size 300
%  \item mean
%\end{itemize}
%\vspace{1cm}
%\end{slide}


%\begin{slide}{Plot statistical charts and see the sale pattern} 
%By using the matplotlib to plot the photoes which describe the sale pattern
%\vspace{1cm}
%\begin{figure}[ht]%插入图片
  %\centering%用于居中
  %\includegraphics[scale=0.9]{E:/tulip-flip/templatex-master/powerdot-tuliplab/logos/0003.eps}
  %\caption{Displaying the sale pattern}%图片标题
  %\end{figure}
  %\vspace{0.5cm}
%From the figures we can know that 2nd store is the topper and 7th store is the least revenue generating one
%\end{slide}

%\begin{slide}{Variation in scale of the sale transacted}
%Displaying the distribution of sales volume 
%\vspace{1cm}
%\begin{figure}[ht]%插入图片
  %\centering%用于居中
  %\includegraphics[scale=1.0]{E:/tulip-flip/templatex-master/powerdot-tuliplab/logos/0004.eps}
  %\caption{sales volume's distribution}%图片标题
  %\end{figure}
  %\vspace{1cm}
  %From this figure we can know that more and more sales volume is belong to [0,50]
%\end{slide}

%\begin{slide}{Store total sales}
%Displaying the total sales of the stores 
%\vspace{1.0cm}
%\begin{figure}[ht]%插入图片
  %\centering%用于居中
  %\includegraphics[scale=1.0]{E:/tulip-flip/templatex-master/powerdot-tuliplab/logos/0005.eps}
  %\caption{The total sales of all stores}%图片标题
  %\end{figure}
  %\vspace{0.5cm}
  %From this figure we can know that 2nd store is the topper of the all stores
%\end{slide}

%\begin{slide}{Item total sales}
  %Displaying the total sales of the items 
  %\vspace{0.8cm}
  %\begin{figure}[ht]%插入图片
    %\centering%用于居中
    %\includegraphics[scale=0.7]{E:/tulip-flip/templatex-master/powerdot-tuliplab/logos/0006.eps}
    %\caption{The total sales of all items}%图片标题
   % \end{figure}
    %\vspace{0.3cm}
   % From this figure we can know the total sales of all items.Obviously,we can know every item's sales
%\end{slide}

%\begin{slide}{All store's performance}
%Displaying the all store's performance over the time
%\vspace{1cm}
%\begin{figure}[ht]%插入图片
  %\centering%用于居中
  %\includegraphics[scale=0.7]{E:/tulip-flip/templatex-master/powerdot-tuliplab/logos/0007.eps}
  %\caption{The performance of all stores}%图片标题
  %\end{figure}
  %From this figure we can know every stores sales changing overtime
%\end{slide}

%\begin{slide}{All item's performance}
  %Displaying the all items' performance over the time
  %\vspace{1.2cm}
  %\begin{figure}[ht]%插入图片
    %\centering%用于居中
    %\includegraphics[scale=0.7]{E:/tulip-flip/templatex-master/powerdot-tuliplab/logos/0008.eps}
    %\caption{The performance of all stores items}%图片标题
    %\end{figure}
    %From this figure we can know every item sales changing overtime 
  %\end{slide}

%\begin{slide}{Individual pattern of store's and item's sale}
%Displaying the performance of the individual score and item
%\vspace{1cm}
%\begin{figure}[ht]%插入图片
  %\centering%用于居中
  %\includegraphics[scale=0.9]{E:/tulip-flip/templatex-master/powerdot-tuliplab/logos/0009.eps}
  %\includegraphics[scale=0.9]{E:/tulip-flip/templatex-master/powerdot-tuliplab/logos/0010.eps}
  %\caption{The performance of the individual store and item}%图片标题
  %\end{figure}
  %\vspace{0.8cm}
  %From this figure we can know individual pattern of item's sale and store's sale
%\end{slide}
\section{Trainning Modele}

\begin{slide}[toc=,bm=]{Modeling}
\begin{itemize}
  \item LSTM\\
  \item  SVM\\
  \item  CNN\\
\end{itemize}
\end{slide}

\begin{slide}[toc=,bm=]{LSTM}
\begin{itemize}
  \hspace{0.5cm}
  Train on 1611 samples, validate on 378 samples\\
Epoch 1/3\\
1611/1611 [==============================] - 50s - loss: 0.6942 - acc: 0.5208 - val_loss: 0.6935 - val_acc: 0.5079\\
Epoch 2/3\\
1611/1611 [==============================] - 51s - loss: 0.6722 - acc: 0.5971 - val_loss: 0.6904 - val_acc: 0.5106\\
Epoch 3/3\\
1611/1611 [==============================] - 50s - loss: 0.5941 - acc: 0.7492 - val_loss: 0.6831 - val_acc: 0.5661\\
378/378 [==============================] - 3s \\    
Test score: 0.683092702633\\
Test accuracy: 0.566137566138\\
prediction accuracy:  0.566137566138\\
\vspace{1cm}
\end{itemize}
\end{slide}

\begin{slide}[toc=,bm=]{CNN}
  \begin{itemize}
    \hspace{0.5cm}
    Train on 1611 samples, validate on 378 samples\\
Epoch 1/1\\
1611/1611 [==============================] - 14s - loss: 0.6900 - acc: 0.5317 - val_loss: 0.6974 - val_acc: 0.5079\\
352/378 [==========================>...] - ETA: 0sTest score: 0.69742725829\\
Test accuracy: 0.507936509829\\
prediction accuracy:  0.507936507937\\
  \vspace{1cm}
  \end{itemize}
  \end{slide}

\begin{slide}[toc=,bm=]{The score of models}
  \begin{table}[htbp]  \centering
    \caption{The score of  models}
    \label{tbl:data information}
    \begin{tabular}{ccccccc}
      \hline
      % after \\: \hline or \cline{col1-col2} \cline{col3-col4} ...
        & model          & score \\
      \hline
      1 &  LSTM          & 0.566137566138  \\
      2 &  CNN           & 0.507936507937  \\
      3 &  svm           & 0.595238095238  \\
     
      \hline 
      %\bottomrule
    \end{tabular}
  \end{table}
\end{slide}

\section{Conclusion}
\begin{slide}[toc=,bm=]{Conclusion}
%\begin{description}
  \item[1] At present, we can only simply use the function call of the algorithm in NLP, and the parameter setting inside the function is still in the initial stage
  \item[2] I need to master a variety of natural language processing methods
  \item[3] The next step is to improve the quality of the algorithm based on the internal principles of the function
%\end{description}
\end{slide}

\section{Thanks for watching}


%\begin{slide}[toc=,bm=]{}
  %\tableofcontents[content=sections]
  %\end{slide}
  %\section{Third section}
  %\begin{slide}[toc=,bm=]{Acknowledge}
  %\tableofcontents[content=currentsection,type=1]
  %\end{slide}
%\begin{slide}{Acknowledge}
%\vspace{3.5cm}
%\centering
%\huge
%\textit{Thank you for watching the slides} 
%\end{slide}
\end{document}

\endinput
