%%
%% This is file `tikzposter-template.tex',
%% generated with the docstrip utility.
%%
%% The original source files were:
%%
%% tikzposter.dtx  (with options: `tikzposter-template.tex')
%%
%% This is a generated file.
%%
%% Copyright (C) 2014 by Pascal Richter, Elena Botoeva, Richard Barnard, and Dirk Surmann
%%
%% This file may be distributed and/or modified under the
%% conditions of the LaTeX Project Public License, either
%% version 2.0 of this license or (at your option) any later
%% version. The latest version of this license is in:
%%
%% http://www.latex-project.org/lppl.txt
%%
%% and version 2.0 or later is part of all distributions of
%% LaTeX version 2013/12/01 or later.
%%


\documentclass{tikzposter} %Options for format can be included here

\usepackage{todonotes}

\usepackage[tikz]{bclogo}
\usepackage{lipsum}
\usepackage{amsmath}

\usepackage{booktabs}
\usepackage{longtable}
\usepackage[absolute]{textpos}
\usepackage[it]{subfigure}
\usepackage{graphicx}
\usepackage{cmbright}
%\usepackage[default]{cantarell}
%\usepackage{avant}
%\usepackage[math]{iwona}
\usepackage[math]{kurier}
\usepackage[T1]{fontenc}


%% add your packages here
\usepackage{hyperref}
% for random text
\usepackage{lipsum}
\usepackage[english]{babel}
\usepackage[pangram]{blindtext}

\colorlet{backgroundcolor}{blue!10}

 % Title, Author, Institute
\title{FLIP01 FINAL PRESENTATION}
\author{Guanzhang Huang}
\institute{Xi'an Shiyou University, China}
%\titlegraphic{logos/tulip-logo.eps}

%Choose Layout
\usetheme{Wave}

%\definebackgroundstyle{samplebackgroundstyle}{
%\draw[inner sep=0pt, line width=0pt, color=red, fill=backgroundcolor!30!black]
%(bottomleft) rectangle (topright);
%}
%
%\colorlet{backgroundcolor}{blue!10}

\begin{document}


\colorlet{blocktitlebgcolor}{blue!23}

 % Title block with title, author, logo, etc.
\maketitle

\begin{columns}
 % FIRST column
\column{0.5}% Width set relative to text width

%%%%%%%%%% -------------------------------------------------------------------- %%%%%%%%%%
 %\block{Main Objectives}{
%  	      	\begin{enumerate}
%  	      	\item Formalise research problem by extending \emph{outlying aspects mining}
%  	      	\item Proposed \emph{GOAM} algorithm is to solve research problem
%  	      	\item Utilise pruning strategies to reduce time complexity
%  	      	\end{enumerate}
%%  	      \end{minipage}
%}
%%%%%%%%%% -------------------------------------------------------------------- %%%%%%%%%%


%%%%%%%%%% -------------------------------------------------------------------- %%%%%%%%%%
\block{Introduction}{
  Using 8 years daily news headlines to predict stock market movement.
  The Kaggle dataset contains date, label, and 28 columns of same-day news data.
The data were divided into a training set and a test set to 
predict the relationship between news information and 
stock market movements on the day.
}
%%%%%%%%%% -------------------------------------------------------------------- %%%%%%%%%%
\block{data information}{
  \begin{description}
    \item [date] - Date information.
    \item [label] - 1 for trading, 0 for falling.
    \item [top x] - News ID.
    
  \end{description} 

  \begin{table}[htbp]  \centering
    \caption{The head of the train data}
    \label{tbl:data information}
    \begin{tabular}{ccccccc}
      \hline
      % after \\: \hline or \cline{col1-col2} \cline{col3-col4} ...
      & Date & label & Top1 & ...\\
      \hline
      0 & 2016-08-08    & 0     & b"Georgia 'downs two Russian warplanes' as cou... & ...\\
      1 & 2016-08-11    & 1     & b'Why wont America and Nato help us? If they w... & ...\\
      2 & 2016-08-12    & 0     & b'Remember that adorable 9-year-old who sang a... & ...\\
      3 & 2016-08-13    & 0     & b' U.S. refuses Israel weapons to attack Iran:... & ...\\
      4 & 2016-08-14    & 1     & b'All the experts admit that we should legalis... & ...\\
      5 & ...           & ...   & ...                                               & ...\\
          \hline 
      %\bottomrule
    \end{tabular}
  \end{table}

  In the table, there are date, label and 25 column top values, 
respectively representing the date, 
whether the stock market rose or fell and 
the news information of the day.
Through the training of the data, 
in order to get the relationship between 
the rise and fall of the stock market and the news of the day.
}
%%%%%%%%%% -------------------------------------------------------------------- %%%%%%%%%%
\block{Text Processing}{
  Text data is data information that cannot be recognized by a computer. 
  Only by digitizing text data can a computer be able to process the data.
  This chapter mainly introduces the processing method of text data.
  
  \begin{itemize}
    \item Get rid of the HTML tag
    \item Remove the punctuation
    \item Cut into the word /token
    \item stopwords
    \item Reorganize into new sentences
    \end{itemize}
}
%%%%%%%%%% -------------------------------------------------------------------- %%%%%%%%%%


%%%%%%%%%% -------------------------------------------------------------------- %%%%%%%%%%

%\note{Note with default behavior}

%\note[targetoffsetx=12cm, targetoffsety=-1cm, angle=20, rotate=25]
%{Note \\ offset and rotated}

 % First column - second block


%%%%%%%%%% -------------------------------------------------------------------- %%%%%%%%%%
\block{Modeling}{
  There are many machine learning methods for text classification. We have selected the following five methods:
  \vspace{1cm}
  \begin{description}
    \begin{itemize}
      \item  Logistic Regression
      \item  KNN
      \item  Random forest
      \item  SVM
      \item  CNN
      \end{itemize}
  \end{description} 
}
%%%%%%%%%% -------------------------------------------------------------------- %%%%%%%%%%


% SECOND column
\column{0.5}
 %Second column with first block's top edge aligned with with previous column's top.

%%%%%%%%%% -------------------------------------------------------------------- %%%%%%%%%%
\block{Word2vec}
\item embedding_size\\ %词向量维度
\item learning_rate\\  %学习率
\item nagetive_sample\\%负采样模型

\
\hspace{0.5cm}  
   \item model[ok]\\
   \vspace{1cm}
   128D:array([-0.29960674,0.03145241...])


\block{model}{
  \begin{itemize}
    \item LSTM\\
    \item  SVM\\
    \item  CNN\\
  \end{itemize}
\
\item LSTM
\begin{itemize}
  \hspace{0.5cm}
  Train on 1611 samples, validate on 378 samples\\
Epoch 1/3\\
1611/1611 [==============================] - 50s - loss: 0.6942 - acc: 0.5208 - val_loss: 0.6935 - val_acc: 0.5079\\
Epoch 2/3\\
1611/1611 [==============================] - 51s - loss: 0.6722 - acc: 0.5971 - val_loss: 0.6904 - val_acc: 0.5106\\
Epoch 3/3\\
1611/1611 [==============================] - 50s - loss: 0.5941 - acc: 0.7492 - val_loss: 0.6831 - val_acc: 0.5661\\
378/378 [==============================] - 3s \\    
Test score: 0.683092702633\\
Test accuracy: 0.566137566138\\
prediction accuracy:  0.566137566138\\
\vspace{1cm}
\end{itemize}
\
\item CNN
 \begin{itemize}
  \hspace{0.5cm}
  Train on 1611 samples, validate on 378 samples\\
Epoch 1/1\\
1611/1611 [==============================] - 14s - loss: 0.6900 - acc: 0.5317 - val_loss: 0.6974 - val_acc: 0.5079\\
352/378 [==========================>...] - ETA: 0sTest score: 0.69742725829\\
Test accuracy: 0.507936509829\\
prediction accuracy:  0.507936507937\\
\vspace{1cm}
\end{itemize}
}
%%%%%%%%%% -------------------------------------------------------------------- %%%%%%%%%%
% Second column - first block


%%%%%%%%%% -------------------------------------------------------------------- %%%%%%%%%%
\block[titleleft]{model score}
{
  Figure of left is the first prediction based on the model. Figure of center
shows ten features with high feature importance. Figure of the right is a new
prediction based on the model to get the results needed for this problem.

\begin{description}
      \item   LSTM           0.566137566138
      \item  CNN           0.507936507937
      \item svm           & 0.595238095238
\end{description}
}
%%%%%%%%%% -------------------------------------------------------------------- %%%%%%%%%%
%\block{CNN}{
%  It can be seen that the model gradually started to stabilize when iterating about 10 times.
%  \begin{center}
%    \includegraphics[width=.5\linewidth]{E:/tulip-flip/flip01/photo/03.eps}
%    \quad\includegraphics[width=.5\linewidth]{E:/tulip-flip/flip01/photo/04.eps}	
%  \end{center}
%}
%\block[titlewidthscale=1, bodywidthscale=1]
%{Experiment and Analysis}
%{
%This time the accuracy is slightly lower. There may be two reasons. The first is to use the average method 
%when converting word vectors into sentence vectors. The other is that word vectors are trained with their own words,
 %and the distance between word vectors is relatively close. So there is no distinction.
%}

% Second column - second block
%%%%%%%%%% -------------------------------------------------------------------- %%%%%%%%%%
\block[titlewidthscale=1, bodywidthscale=1]
{Conclusion}
{
  \begin{description}
  \item [1] I need to master a variety of natural language processing methods\\
  \item [2] The next step is to improve the quality of the algorithm based on the internal principles of the function
  \end{description}
}
%%%%%%%%%% -------------------------------------------------------------------- %%%%%%%%%%


% Bottomblock
%%%%%%%%%% -------------------------------------------------------------------- %%%%%%%%%%
\colorlet{notebgcolor}{blue!20}
\colorlet{notefrcolor}{blue!20}
\note[targetoffsetx=8cm, targetoffsety=-4cm, angle=30, rotate=15,
radius=2cm, width=.26\textwidth]{
Acknowledgement
\begin{itemize}
    \item
    Thank you
 \end{itemize}
}

%\note[targetoffsetx=8cm, targetoffsety=-10cm,rotate=0,angle=180,radius=8cm,width=.46\textwidth,innersep=.1cm]{
%Acknowledgement
%}

%\block[titlewidthscale=0.9, bodywidthscale=0.9]
%{Acknowledgement}{
%}
%%%%%%%%%% -------------------------------------------------------------------- %%%%%%%%%%

\end{columns}


%%%%%%%%%% -------------------------------------------------------------------- %%%%%%%%%%
%[titleleft, titleoffsetx=2em, titleoffsety=1em, bodyoffsetx=2em,%
%roundedcorners=10, linewidth=0mm, titlewidthscale=0.7,%
%bodywidthscale=0.9, titlecenter]

%\colorlet{noteframecolor}{blue!20}
\colorlet{notebgcolor}{blue!20}
\colorlet{notefrcolor}{blue!20}
\note[targetoffsetx=-13cm, targetoffsety=-12cm,rotate=0,angle=180,radius=8cm,width=.96\textwidth,innersep=.4cm]
{
\begin{minipage}{0.3\linewidth}
\centering
\includegraphics[width=24cm]{logos/tulip-wordmark.eps}
\end{minipage}
\begin{minipage}{0.7\linewidth}
{ \centering
  FLIP01 FINAL PRESENTATION
  23/2/2020, Xi'an, China
}
\end{minipage}
}
%%%%%%%%%% -------------------------------------------------------------------- %%%%%%%%%%


\end{document}

%\endinput
%%
%% End of file `tikzposter-template.tex'.
